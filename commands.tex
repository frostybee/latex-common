%--------------------------------------
%--- Summary of features/commands:
%-- Colored notes in PDF.
%-- parametrized list environments.
%-- colored shapes.
%--- custom text hightlight.
%-------------------
%%\pdfoutput=1

\usepackage{url}
% Including fullpage will disable the running title.
% \usepackage{fullpage}
\usepackage{multicol}[2]
%\usepackage{makeidx}
\usepackage{graphicx}
%\usepackage{listings}
\usepackage{multirow}
\usepackage{enumitem}
\usepackage[usenames,dvipsnames]{color}
\usepackage{xcolor}
\usepackage{verbatim} % used to display code
\usepackage{fancyvrb}
\usepackage{amsthm}
\usepackage{listings}
\usepackage{caption}
\usepackage{acronym}
%\usepackage{amsthm}
% bbding documentation, @see: ftp://ftp.dante.de/tex-archive/fonts/bbding/bbding.pdf
\usepackage{bbding}
\usepackage{fancyhdr, lastpage, pmboxdraw}
\usepackage[draft]{todonotes}   % notes showed
%\usepackage[colorlinks,pagebackref,pdfusetitle,urlcolor=red,citecolor=darkblue,linkcolor=blue,bookmarksnumbered,plainpages=false]{hyperref}
% for highlighting text.
\usepackage{soul}
\usepackage{vhistory}
%\pagestyle{empty}
%-- @see http://ctan.sharelatex.com/tex-archive/fonts/fontawesome/doc/fontawesome.pdf
\usepackage{fontawesome}

% http://ctan.math.washington.edu/tex-archive/macros/latex/contrib/xargs/xargs.pdf
\usepackage{xargs}
%-- Uncomment this if you want to reduce the space gap between sections' title and text.
%\usepackage[compact]{titlesec}  

\hypersetup{
    colorlinks,
    linkcolor=blue,
    urlcolor=blue,
    citecolor=red,
    bookmarks, linktoc=all		
}


%----------------------------------------------------------------------------------
%-----	Cross-reference commands.
%----------------------------------------------------------------------------------
\newcommand{\xf}[1]{Figure~\ref{#1}}
\newcommand{\xp}[1]{page~\pageref{#1}}
\newcommand{\xs}[1]{Section~\ref{#1}}
\newcommand{\xa}[1]{Appendix~\ref{#1}}
\newcommand{\xc}[1]{Chapter~\ref{#1}}
\newcommand{\xt}[1]{Table~\ref{#1}}
\newcommand{\xl}[1]{Listing~\ref{#1}}
\newcommand{\xr}[1]{\cite{#1}}
%
\newcommand{\file}[1]{\url{#1}\index{Files!#1}}
\newcommand{\tool}[1]{\texttt{#1}\index{Tools!#1}}
\newcommand{\option}[1]{\texttt{#1}\index{Options!#1}}
\newcommand{\api}[1]{\texttt{#1}\index{API!#1}}
\newcommand{\apipackage}[1]{\url{#1}\index{API!Packages!#1}\index{Packages!#1}}
\newcommand{\datatype}[1]{\texttt{#1}\index{Type!#1}}
\newcommand{\codesegment}[1]{\texttt{\##1}\index{Segments!\##1}}

%----------------------------------------------------------------------------------
%-----@spacing	Spacing Commands.
%----------------------------------------------------------------------------------
\newcommand{\vspbpara}{\vspace*{.09in}}    
\newcommand{\customvspace}{\vspace{.5cm}}    
%\titlespacing{\section}{0pt}{12pt}{9pt}

%----------------------------------------------------------------------------------
%-----@lists	Lists-related Commands.
%----------------------------------------------------------------------------------
% Disable spaces between list items
% E.g.,: #1: spacing option, [noitemsep], etc.
\newcommand{\beelistspacesettings}[1]{
\setlist{#1} 
}    
%------------- Custom list with adjustable space between list items --------
%-----
% #1: list items
% #2: list type [itemize|enumerate]
% #3: space size, e.g., 8pt
% #4: label character, e.g.,: --
\newcommand{\beelist}[4]
{
    \begin{#2}[label=#4, itemsep=#3,parsep=0pt,topsep=1pt,partopsep=4pt]
           #1  % <-- list content
    \end{#2}
}
%---- List enum options ---- 
%-- Numbering style.
\newcommand{\beelstarabic}{\arabic*}
\newcommand{\beelstalpha}{\alph*}

%-----
%----------------------------------------------------------------------------------


%----------------------------------------------------------------------------------
%-----@section:			Section Commands.
%----------------------------------------------------------------------------------
% Creates an underlined non-numbered section header.
\newcommand{\customsection}[1]{\section*{\underline{#1}}}


%----------------------------------------------------------------------------------
%-------- TODO notes and comments in LaTeX documents ------ 
%----------------------------------------------------------------------------------

%-- Notes commands
% Select what to do with command \comment:  
% \newcommand{\comment}[1]{}  %comment not showed
% e.g,: \comment{This is a simple comment between text} 
% \todo{This is a to do note at margin}
\newcommand{\beecomment}[1]
{
   \par {\bfseries \color{blue} #1 \par}
} %comment showed
%--
%-------- : Different notes colors: --------------------------------------------------------------
%--  TODO NOTES related colors.
\newcommandx{\unsure}[2][1=]{\todo[linecolor=red,backgroundcolor=red!25,bordercolor=red,#1]{#2}}
\newcommandx{\change}[2][1=]{\todo[linecolor=blue,backgroundcolor=blue!25,bordercolor=blue,#1]{#2}}
\newcommandx{\info}[2][1=]{\todo[linecolor=OliveGreen,backgroundcolor=OliveGreen!25,bordercolor=OliveGreen,#1]{#2}}
\newcommandx{\improvement}[2][1=]{\todo[linecolor=Plum,backgroundcolor=Plum!25,bordercolor=Plum,#1]{#2}}
%-- you can use this command if whether you want to show (toogle) the notes in the resulting document or not
% by changing remove the ''disable'' option below. 
% Example: \pinnabletodo{This is hidden since option `disable' is chosen!}
\newcommandx{\pinnabletodo}[2][1=]{\todo[disable,#1]{#2}}

% Custom inline todo notes.
\newcommand{\inlinetodo}[1]{\todo[inline]{#1}}


\newcommand{\vtitlespacing}{\vskip 0.1cm}
%\newcommand{\paragraphentry}[1]{\noindent \textbf{\Large \underline{#1}} }


%----------------------------------------------------------------------------------
%-----@color:		Custom colors definitions, text and background colors utils.
%----------------------------------------------------------------------------------
%--  more colors codes can be found here: http://latexcolor.com/
%-- usage: {\color{declared-color} some text}. e.g.,: {\color{darkblue}{ This text will appear darkblue-colored}}
%-
\definecolor{darkestblue}{rgb}{0.0, 0.0, 0.55}
\definecolor{darkblue}{rgb}{0,0,.6}
\definecolor{darkred}{rgb}{.7,0,0}
\definecolor{darkgreen}{rgb}{0,.6,0}
\definecolor{darkestred}{rgb}{.8,.1,0}
\definecolor{red}{rgb}{.98,0,0}
\definecolor{OliveGreen}{cmyk}{0.64,0,0.95,0.40}
\definecolor{CadetBlue}{cmyk}{0.62,0.57,0.23,0}
\definecolor{lightlightgray}{gray}{0.93}
\definecolor{darkorange}{rgb}{255,140,0}
\definecolor{fluorescentyellow}{rgb}{0.8, 1.0, 0.0}
\definecolor{darkyellow}{rgb}{1,1,0.34}
\definecolor{lightyellow}{rgb}{1,1,0.6}
\definecolor{coolblack}{rgb}{0.0, 0.18, 0.39}
\definecolor{lightgray}{rgb}{.9,.9,.9}
\definecolor{darkgray}{rgb}{.4,.4,.4}
\definecolor{purple}{rgb}{0.65, 0.12, 0.82}

%--- Highlighting text.
%- background color (highlight) command.
\newcommand{\highlight}[1]{\hl{#1}}

%- Highlight text with a custom color.
%- usage:  \mycustomhl{declared-color}{some text}
\newcommand{\mycustomhl}[2]{{\sethlcolor{#1}\hl{#2}}}

%----------------------------------------------------------------------------------
%-----@shapes	My custom-colored shapes.
%----------------------------------------------------------------------------------

%-- Creates a custom-colored shape
%-- usage: {\mycustomshape{declared-color}{shape-command}{spacing-command} 
%-- e.g.,: {\mycustomshape{darkblue}{\Rectangle}{~}. 
\newcommand{\mycustomshape}[3]
{
    {\color{#1}{#2}}{#3}  
}


%----------------------------------------------------------------------------------
%-----@listing	Source Code Listing.
%----------------------------------------------------------------------------------
 \DeclareCaptionFont{white}{\color{white}}
 \DeclareCaptionFormat{listing}{\colorbox{gray}{\parbox{\textwidth}{#1#2#3}}}
 \captionsetup[lstlisting]{format=listing,labelfont=white,textfont=white}
 
 \VerbatimFootnotes % Required, otherwise verbatim does not work in footnotes!

%
% Command for adjuctiong tabled row header (left, center or right)
% Usage example: \thead{<Header text 1>} & \thead{<Header 2>} & \thead{<Header 3>} & \thead{<Header 4>} 
\newcommand*{\thead}[1]{\multicolumn{1}{l}{\bfseries #1}}


\newcommand{\sourcefloat}[3]
{
	\begin{figure}[!hp]
		\begin{centering}
			\begin{minipage}{0.5\textwidth}
				\source{#1}
			\end{minipage}
			\caption{\small{#3}}
			\label{#2}
		\end{centering}
	\end{figure}
}

\newcommand{\mytodo}[0]
{
	{\Large \[TODO\]}
}

%----------------------------------------------------------------------------------
%-----@utlis 	A bunch of utilities .
%----------------------------------------------------------------------------------

%- Bold & italic text.
\newcommand{\mytbi}[1]{\textbf{\textit{#1}}}


%-- Specical characters.
\newcommand{\dashnospace}[0]{\--}
\newcommand{\dashwithspace}[0]{\-- }
\newcommand{\tab}[1]{\hspace*{#1}}
\newcommand{\shrule}[0]{\vspace{3pt}\hrule\vspace{6pt}}
\newcommand{\ehrule}[0]{\vspace{6pt}\hrule\vspace{3pt}}
\newcommand{\source}[1]
{
    {\shrule}
    \scriptsize
    #1
    \normalsize
    \hrule
}


%
% Def
%
\newcommand{\statement}[2]
{
    \vspace{7pt}
    \shrule
    {\bf #1}
    
    #2
    \ehrule
    \vspace{7pt}
}

% \newcommand{\proposition}[2]
\newcommand{\sproposition}[2]
{
    \statement{Proposition #1}{#2}
}

% \newcommand{\definition}[2]
\newcommand{\sdefinition}[2]
{
    \statement{Definition #1}{#2}
}

% \newcommand{\axiom}[2]
\newcommand{\saxiom}[2]
{
    \statement{Axiom #1}{#2}
}

% \newcommand{\theorem}[2]
\newcommand{\stheorem}[2]
{
    \statement{Theorem #1}{#2}
}

%----------------------------------------------------------------------------------
%-----@listing 	Various language-related listing configurations.
%----------------------------------------------------------------------------------
%-- JavaScript
%- @see: https://tex.stackexchange.com/questions/89574/language-option-supported-in-listings
\lstdefinelanguage{JavaScript}{
    keywords={break, case, catch, continue, debugger, default, delete, do, else, false, finally, for, function, if, in, instanceof, new, null, return, switch, this, throw, true, try, typeof, var, void, while, with},
    morecomment=[l]{//},
    morecomment=[s]{/*}{*/},
    morestring=[b]',
    morestring=[b]",
    ndkeywords={class, export, boolean, throw, implements, import, this},
    keywordstyle=\color{blue}\bfseries,
    ndkeywordstyle=\color{darkgray}\bfseries,
    identifierstyle=\color{black},
    commentstyle=\color{darkgreen}\ttfamily,
    stringstyle=\color{red}\ttfamily,
    sensitive=true
}
%-- Command to set the JavaScript language settings.
\newcommand{\lstsetjavascript}{
    \lstset{
        language=JavaScript,
        backgroundcolor=\color{lightgray},
        extendedchars=true,
        basicstyle=\footnotesize\ttfamily,
        showstringspaces=false,
        showspaces=false,
        numbers=left,
        numberstyle=\footnotesize,
        numbersep=9pt,
        tabsize=2,
        breaklines=true,
        showtabs=false,
        captionpos=b
    }
}

\newcommand{\lstsetbash}{
    \lstset{
        language=bash,                          % Code langugage
        basicstyle=\ttfamily,                   % Code font, Examples: \footnotesize, \ttfamily
        keywordstyle=\color{OliveGreen},        % Keywords font ('*' = uppercase)
        commentstyle=\color{gray},              % Comments font
        numbers=left,                           % Line nums position
        numberstyle=\tiny,                      % Line-numbers fonts
        stepnumber=1,                           % Step between two line-numbers
        numbersep=5pt,                          % How far are line-numbers from code
        backgroundcolor=\color{lightlightgray}, % Choose background color
        frame=none,                             % A frame around the code
        tabsize=2,                              % Default tab size
        captionpos=t,                           % Caption-position = bottom
        breaklines=true,                        % Automatic line breaking?
        breakatwhitespace=false,                % Automatic breaks only at whitespace?
        showspaces=false,                       % Dont make spaces visible
        showtabs=false,                         % Dont make tabls visible
        columns=flexible,                       % Column format
        morekeywords={__global__, __device__},  % CUDA specific keywords
    }
}


